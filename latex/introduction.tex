\section*{Introduction}
\addcontentsline{toc}{section}{Introduction}

\sffamily

The latest version of the integer book can be downloaded from: \url{https://jessevoostrum.github.io/music-visualisation/integer-book.pdf}

\subsubsection*{the method}
The songs in this book are represented using numbers. Each number refers to a location within the major scale. For example, if the song is in D major, the 3 would refer to the F\#. The numbers in the melody refer to single tones. The numbers of the chords refer to the chord corresponding to the tone. Again, in D major, the 6- chord would be B-D-F\#.  

\subsubsection*{the name}
A Real Book is a compilation of lead sheets for jazz standards. The name is derived from "fake books", so called because they contained only rough outlines of music pieces rather than fully notated scores.\footnote{source: \url{https://en.wikipedia.org/wiki/Real_Book}} In mathematics the real numbers are a collection containing all numbers with a possibly infinite decimal expansion. A subset of these are the integer numbers, which can be written without the use of decimals. Since these are the only numbers we use to represent the music in this book, it is called The Integer Book.

\subsubsection*{the code}
The code to produce these visualisations can be viewed on: \url{https://github.com/jessevoostrum/music-visualisation}. If you would like to contribute to the code or have questions, you can send me an email: jessevoostrum@gmail.com.

\subsubsection*{donate}
If you feel that you are benefitting from these visualisations and would like to show your appreciation through a financial donation, this can be done via \href{https://www.paypal.com/cgi-bin/webscr?cmd=_donations&business=jessevoostrum@gmail.com&no_shipping=1&no_note=1&tax=0&currency_code=EUR&lc=US&bn=PP_DonationsBF}{\underline{this link}}. 