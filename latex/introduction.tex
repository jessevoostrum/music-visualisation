
% \renewcommand\sfdefault{phv}
% \normalfont

\section*{Introduction}
\addcontentsline{toc}{section}{Introduction}

% \sffamily

The latest version of the integer book can be downloaded from: \url{https://jessevoostrum.github.io/music-visualisation/integer-book.pdf}

\subsubsection*{the philosophy}
\textit{“Understand the music you hear, play the music you imagine”} \\

\noindent The above slogan\footnote{This is the slogan from the musical education platform “Improvise for Real”. I can highly recommend checking out their resources at \url{https://improviseforreal.com/}} summarises succinctly my personal goal in my musical journey. In order to accomplish this goal, we need an easy way to translate the feeling that music generates when we hear or imagine it, to theoretical symbols we can use to understand and play the music. \\

\noindent The problem with conventional symbols like F\# or C is that they don't correspond to one particular feeling, since an F\# will generate a very different feeling in a melody in D major (happy / uplifting) compared to a melody in G major (tension). However, an F\# in G major does generate a very similar feeling to a B in C major. This is because they are both the 7th of their respective scales.  This holds true for any relative position in the scale: e.g. E in A major and G\# in C\# major (both the 5th) or Eb in C major and Bb in G major (both the b3th). \\

\noindent Therefore, if we think about and write down melodies in terms of their relative position in the scale, it gets very easy to translate the feeling the music generates to their corresponding theoretical symbol (a number). And together with the key of the song, this also gives us all the information we need to find the right note on our instrument. 

\subsubsection*{the method}
The songs in this book are represented using numbers. Each number refers to a location within the major scale. For example, if the song is in D major, the 3 would refer to the F\#. The numbers in the melody refer to single tones. The numbers of the chords refer to the chord corresponding to the tone. Again, in D major, the 6- chord would be B-D-F\#.  

\subsubsection*{the name}
A Real Book is a compilation of lead sheets for jazz standards. The name is derived from "fake books", so called because they contained only rough outlines of music pieces rather than fully notated scores.\footnote{source: \url{https://en.wikipedia.org/wiki/Real_Book}} In mathematics the real numbers are a collection containing all numbers with a possibly infinite decimal expansion. A subset of these are the integer numbers, which can be written without the use of decimals. Since these are the only numbers we use to represent the music in this book, it is called The Integer Book.

\subsubsection*{the code}
The code to produce these visualisations can be viewed on: \url{https://github.com/jessevoostrum/music-visualisation}. If you would like to contribute to the code or have questions, you can send me an email: jessevoostrum@gmail.com.

\subsubsection*{donate}
If you feel that you are benefitting from these visualisations and would like to show your appreciation through a financial donation, this can be done via \href{https://www.paypal.com/cgi-bin/webscr?cmd=_donations&business=jessevoostrum@gmail.com&no_shipping=1&no_note=1&tax=0&currency_code=EUR&lc=US&bn=PP_DonationsBF}{\underline{this link}}. 